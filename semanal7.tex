\documentclass{article}

\usepackage[margin=2cm]{geometry}
\usepackage[T1]{fontenc}
\usepackage[parfill]{parskip}

\usepackage{listings}
\usepackage[dvipsnames]{xcolor}
\usepackage{minted}

\lstdefinestyle{minilisp}{
  commentstyle=\color{codegreen},
  keywordstyle=\color{OrangeRed},
  basicstyle=\ttfamily,
  identifierstyle=\color{Cyan}
}

\lstdefinelanguage{Minilisp}{
  language=Lisp,
  morekeywords={lambda,if0}
}

\lstset{style=minilisp,language=Minilisp,mathescape}
\lstMakeShortInline[columns=fixed]@

\begin{document}
  \begin{center}
    {\Large Evaluación semanal 7}\\
    {Santiago González Tamariz}\\
    {\small Lenguajes de Programación}
  \end{center}

  \begin{enumerate}
    \item Dada la siguiente expresión en \textbf{MiniLisp}
      \begin{lstlisting}
(let (sum (lambda (n) (if0 n 0 (+ n (sum (- n 1))))))
   (sum 5))
      \end{lstlisting}
      \begin{itemize}
        \item Ejecutarla y explicar el resultado.

          Paso 1:
          \begin{lstlisting}
(let (sum (lambda (n) (if0 n 0 (+ n (sum (- n 1))))))
  (sum 5))
          \end{lstlisting}
          $\epsilon = $ \begin{tabular}{|c|c|}
            @sum@ & $\langle$@n@, @(if0 n 0 (+ n (sum (- n 1))))@, $\epsilon_1\rangle$\\
            \hline
          \end{tabular}
          \hspace{0.5cm} $\epsilon_1 = $ \begin{tabular}{|c|c|}
             & \\
            \hline
          \end{tabular}\\
          Añadimos la cerradura de @sum@ al ambiente y creamos el subambiente correspondiente

          Paso 2:
          \begin{lstlisting}
(sum 5)
          \end{lstlisting}
          $\epsilon = $ \begin{tabular}{|c|c|}
            $\Rightarrow$@sum@ & $\langle$@n@, @(if0 n 0 (+ n (sum (- n 1))))@, $\epsilon_1\rangle$\\
            \hline
          \end{tabular}
          \hspace{0.5cm} $\epsilon_1 = $ \begin{tabular}{|c|c|}
             @n@ & @5@ \\
            \hline
          \end{tabular}\\
          Es necesario evaluar @sum@ para poder continuar

          Paso 3:
          \begin{lstlisting}
(if0 n 0 (+ n (sum (- n 1))))$, \epsilon_1$
          \end{lstlisting}
          $\epsilon = $ \begin{tabular}{|c|c|}
            @sum@ & $\langle$@n@, @(if0 n 0 (+ n (sum (- n 1))))@, $\epsilon_1\rangle$\\
            \hline
          \end{tabular}
          \hspace{0.5cm} $\epsilon_1 = $ \begin{tabular}{|c|c|}
             $\Rightarrow$@n@ & @5@ \\
            \hline
          \end{tabular}\\
          Es necesario evaluar la condición del @if0@ para continuar

          Paso 4:
          \begin{lstlisting}
(if0 5 0 (+ n (sum (- n 1))))$, \epsilon_1$
          \end{lstlisting}
          $\epsilon = $ \begin{tabular}{|c|c|}
            @sum@ & $\langle$@n@, @(if0 n 0 (+ n (sum (- n 1))))@, $\epsilon_1\rangle$\\
            \hline
          \end{tabular}
          \hspace{0.5cm} $\epsilon_1 = $ \begin{tabular}{|c|c|}
             @n@ & @5@ \\
            \hline
          \end{tabular}\\
          Tomamos el else

          Paso 5:
          \begin{lstlisting}
(+ 5 (sum (- n 1)))$, \epsilon_1$
          \end{lstlisting}
          $\epsilon = $ \begin{tabular}{|c|c|}
            @sum@ & $\langle$@n@, @(if0 n 0 (+ n (sum (- n 1))))@, $\epsilon_1\rangle$\\
            \hline
          \end{tabular}
          \hspace{0.5cm} $\epsilon_1 = $ \begin{tabular}{|c|c|}
             $\Rightarrow$@n@ & @5@ \\
            \hline
          \end{tabular}\\
          Es necesario evaluar la suma, iniciamos por el lado izquierdo

          Paso 6:
          \begin{lstlisting}
(+ 5 (sum (- n 1)))$, \epsilon_1$
> error: variable libre
          \end{lstlisting}
          $\epsilon = $ \begin{tabular}{|c|c|}
            @sum@ & $\langle$@n@, @(if0 n 0 (+ n (sum (- n 1))))@, $\epsilon_1\rangle$\\
            \hline
          \end{tabular}
          \hspace{0.5cm} $\epsilon_1 = $ \begin{tabular}{|c|c|}
             @n@ & @5@ \\
            \hline
          \end{tabular}\\
          Al buscar @sum@ en el subambiente actual vemos que no está definido por lo que hay un error de variable libre

        \item Modificarla usando el combinador de punto fijo $Y$, volver a ejecutarla y explicar el resultado.

          Recordemos que $Y =_{def} \lambda f.(\lambda x.f(xx))(\lambda x.f(xx))$ y que $(Y \ fun) a = fun \ (Y \ fun) \ a$, tambien hay que añadirle un
          parámetro extra a nuestra función.
          $$sum = \lambda f.\lambda n.(if0 \ n \ 0 \ (+ \ n \ (f \ (- \ n \ 1))))$$
          Entonces aplicando el combinador de punto fijo nos queda
          $$((Y \ sum) \ 5) = sum \ (Y \ sum) \ 5 = (\lambda f.\lambda n.(if0 \ n \ 0 \ (+ \ n \ (f \ (- \ n \ 1))))) \ (Y \ sum) \ 5$$
          $$= (\lambda n.(if0 \ n \ 0 \ (+ \ n \ ((Y \ sum) \ (- \ n \ 1))))) \ 5 = (if0 \ 5 \ 0 \ (+ \ 5 \ ((Y \ sum) \ (- \ 5 \ 1))))$$
          $$= (+ \ 5 \ (sum \ (Y \ sum) \ 4)) = (+ \ 5 \ ((\lambda f.\lambda n.(if0 \ n \ 0 \ (+ \ n \ (f \ (- \ n \ 1))))) \ (Y \ sum) \ 4))$$
          $$= (+ \ 5 \ (if0 \ 4 \ 0 \ (+ \ 4 \ ((Y \ sum) \ (- \ 4 \ 1))))) = (+ \ 5 \ (+ \ 4 \ (sum \ (Y \ sum) \ 3)))$$
          $$= (+ \ 5 \ (+ \ 4 \ ((\lambda f.\lambda n.(if0 \ n \ 0 \ (+ \ n \ (f \ (- \ n \ 1))))) \ (Y \ sum) \ 3)))$$
          $$= (+ \ 5 \ (+ \ 4 \ (if0 \ 3 \ 0 \ (+ \ 3 \ ((Y \ sum) \ (- \ 3 \ 1)))))) = (+ \ 5 \ (+ \ 4 \ (+ \ 3 \ (sum \ (Y \ sum) \ 2))))$$
          $$= (+ \ 5 \ (+ \ 4 \ (+ \ 3 \ ((\lambda f.\lambda n.(if0 \ n \ 0 \ (+ \ n \ (f \ (- \ n \ 1))))) \ (Y \ sum) \ 2))))$$
          $$= (+ \ 5 \ (+ \ 4 \ (+ \ 3 \ (if0 \ 2 \ 0 \ (+ \ 2 \ ((Y \ sum) \ (- \ 2 \ 1))))))) = (+ \ 5 \ (+ \ 4 \ (+ \ 3 \ (+ \ 2 \ (sum \ (Y \ sum) \ 1)))))$$
          $$= (+ \ 5 \ (+ \ 4 \ (+ \ 3 \ (+ \ 2 \ ((\lambda f.\lambda n.(if0 \ n \ 0 \ (+ \ n \ (f \ (- \ n \ 1))))) \ (Y \ sum) \ 1)))))$$
          $$= (+ \ 5 \ (+ \ 4 \ (+ \ 3 \ (+ \ 2 \ (if0 \ 1 \ 0 \ (+ \ 1 \ ((Y \ sum) \ (- \ 1 \ 1)))))))) = (+ \ 5 \ (+ \ 4 \ (+ \ 3 \ (+ \ 2 \ (+ \ 1 \ (sum \ (Y \ sum) \ 0))))))$$
          $$= (+ \ 5 \ (+ \ 4 \ (+ \ 3 \ (+ \ 2 \ (+ \ 1 \ ((\lambda f.\lambda n.(if0 \ n \ 0 \ (+ \ n \ (f \ (- \ n \ 1))))) \ (Y \ sum) \ 0))))))$$
          $$= (+ \ 5 \ (+ \ 4 \ (+ \ 3 \ (+ \ 2 \ (+ \ 1 \ (if0 \ 0 \ 0 \ (+ \ 0 \ ((Y \ sum) \ (- \ 0 \ 1))))))))) = (+ \ 5 \ (+ \ 4 \ (+ \ 3 \ (+ \ 2 \ (+ \ 1 \ 0)))))$$
          $$= (+ \ 5 \ (+ \ 4 \ (+ \ 3 \ (+ \ 2 \ 1)))) = (+ \ 5 \ (+ \ 4 \ (+ \ 3 \ 3))) = (+ \ 5 \ (+ \ 4 \ 6)) = (+ \ 5 \ 10) = 15$$

      \end{itemize}

    \item Evaluar la siguiente expresión en \textbf{Racket}, explicar su resultado y dar la continuación asociada
      a evaluar usando la notación $\lambda_\uparrow$.
      \begin{minted}[style=autumn]{racket}
> (define c #f)
> (+ 1 (+ 2 (+ 3 (+ (let/cc k (set! c k) 4) 5))))
> (c 10)
      \end{minted}

      Primero se define la variable \texttt{c} como falso, luego tenemos una suma, entonces evaluamos la suma
      iniciando por lo más anidado la parte izquierda. El \texttt{let/cc} nos devuelve 4, entonces ya podemos hacer la suma.
      \begin{minted}[style=autumn]{racket}
> (+ 1 (+ 2 (+ 3 (+ 4 5))))
> (+ 1 (+ 2 (+ 3 9)))
> (+ 1 (+ 2 12))
> (+ 1 14)
> 15
      \end{minted}

      También tenemos que guarda la continuación en \texttt{c}
      $$(\lambda_\uparrow(v) \ (+ \ 1 \ (+ \ 2 \ (+ \ 3 \ (+ \ v \ 5)))))$$

      Entonces al evaluar \texttt{c 10} nos queda
      \begin{minted}[style=autumn,mathescape,escapeinside=||]{racket}
> (c 10)
> ((|$\lambda_\uparrow(v)$| (+ 1 (+ 2 (+ 3 (+ v 5))))) 10)
> (+ 1 (+ 2 (+ 3 (+ 10 5))))
> (+ 1 (+ 2 (+ 3 15)))
> (+ 1 (+ 2 18))
> (+ 1 20)
> 21
      \end{minted}

\clearpage
    \item Realizar los siguientes ejercicios en \textbf{Haskell}:
      \begin{itemize}
        \item Definir la función recurisva \texttt{ocurrenciasElementos} que toma como argumentos dos listas y devuelve una
          lista de parejas, en donde cada pareja contiene en su parte izquierda un elemento de la segunda lista y
          en su parte derecha el número de veces que aparece dicho elemento en la primera lista. Por ejemplo:
          \begin{minted}{haskell}
> ocurrenciasElementos [1,3,6,2,4,7,3,9,7] [5,2,3]
[(5,0),(2,1),(3,2)]
          \end{minted}

          Asumiendo que la segunda lista siempre tendrá un tamaño pequeño, podemos recorrer para cada uno de sus elementos
          a la primera lista e ir contando cuantas veces aparece. En caso de que la segunda lista tenga un tamaño parecido
          a la primera lista nos conviene más ordenar las listas e ir checando cuantos elementos contiguos del mismo hay y si
          cambia entonces pasar a contar las del sigiente elemento.

          \inputminted[style=pastie,fontsize=\small]{haskell}{ej3_1.hs}

        \item Mostrar los registros de activación generados por la función definida en el ejercicio anterior con la llamada
          \texttt{ocurrenciasElementos [1,2,3] [1,2]}.
          \begin{minted}[style=pastie,fontsize=\small]{haskell}
ocurrenciasElementos [1,2,3] [1,2]
= oe [1,2,3] [1,2]
= (1, ct 1 [1,2,3]) : oe [1,2,3] [2]
= (1, 1 + ct 1 [2,3]) : oe [1,2,3] [2]
= (1, 1 + ct 1 [3]) : oe [1,2,3] [2]
= (1, 1 + ct 1 []) : oe [1,2,3] [2]
= (1, 1 + 0) : oe [1,2,3] [2]
= (1,1) : oe [1,2,3] [2]
= (1,1) : (2, ct 2 [1,2,3]) : oe [1,2,3] []
= (1,1) : (2, ct 2 [2,3]) : oe [1,2,3] []
= (1,1) : (2, 1 + ct 2 [3]) : oe [1,2,3] []
= (1,1) : (2, 1 + ct 2 []) : oe [1,2,3] []
= (1,1) : (2, 1 + 0) : oe [1,2,3] []
= (1,1) : (2, 1) : oe [1,2,3] []
= (1,1) : (2, 1) : oe [1,2,3]
= (1,1) : (2, 1) : []
= [(1,1), (2, 1)]
          \end{minted}

        \item Optimizar la función definida usando recursión de cola. Deben transformar todas las funciones auxiliares que utilicen.
          \inputminted[style=pastie,fontsize=\small]{haskell}{ej3_2.hs}

        \item Mostrar los registros de activación generados por la versión de cola con la misma llamada.
          \begin{minted}[style=pastie,fontsize=\small]{haskell}
ocurrenciasElementos [1,2,3] [1,2]
= oe [1,2,3] [1,2] []
= oe [1,2,3] [2] ([] ++ [(1, ct 1 [1,2,3] 0)])
= oe [1,2,3] [2] ([] ++ [(1, ct 1 [2,3] 1)])
= oe [1,2,3] [2] ([] ++ [(1, ct 1 [3] 1)])
= oe [1,2,3] [2] ([] ++ [(1, ct 1 [] 1)])
= oe [1,2,3] [2] ([] ++ [(1, 1)])
= oe [1,2,3] [2] [(1, 1)]
= oe [1,2,3] [] ([(1, 1)] ++ [(2, ct 2 [1,2,3] 0)])
= oe [1,2,3] [] ([(1, 1)] ++ [(2, ct 2 [2,3] 0)])
= oe [1,2,3] [] ([(1, 1)] ++ [(2, ct 2 [3] 1)])
= oe [1,2,3] [] ([(1, 1)] ++ [(2, ct 2 [] 1)])
= oe [1,2,3] [] ([(1, 1)] ++ [(2, 1)])
= oe [1,2,3] [] [(1, 1), (2, 1)]
= [(1, 1), (2, 1)]
          \end{minted}

      \end{itemize}
  \end{enumerate}
\end{document}
